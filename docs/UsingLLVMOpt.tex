\section{What is opt?}

\begin{lstlisting}
 opt .bc -> .bc modular optimizer and analysis printer
\end{lstlisting}

\paragraph{u1}
opt is a commandline tool designed for managing various LLVM passes. It takes
LLVM IR as input and performs various analysis
and optimizations and outputs the transformed IR. The LLVM optimizer is
completely optimizer and can be configured to use a particular optimization at
a particular instant of time. The order of various optimizations or analysis can
be configured to suit the need.

\subsection{Using opt}

If you have LLVM installed on your system, then you can use opt right from the commandline,

\begin{lstlisting}
opt -help
\end{lstlisting}

\paragraph{u2}
To print a .bc file into human readable output,

\begin{lstlisting}
opt -S < a.bc 
\end{lstlisting}

Running a builtin pass,

\begin{lstlisting}
opt -{passname} a.bc -o output.bc
\end{lstlisting}

where `passname` can be one of several passes listed in `opt --help`.

\paragraph{u3}
Loading and running a dynamic pass,
Suppose your pass is libHello.so, so you can load and run this pass as

\begin{lstlisting}
opt -load /path/to/libHello.so -hello a.bc -o output.bc
\end{lstlisting}

If your pass just runs analysis over the IR and you want to print the results,
then you can use `-analyze` option. Note that you have to override the `print`
method of your respective pass class.

\begin{lstlisting}
opt -load /path/to/libHello.so -hello -analyze a.bc -S
\end{lstlisting}

Running multiple passes,
You can run multiple passes by specifying the option for each pass in a order.
The order in which the options are specified is the order in which they will be
executed.

\begin{lstlisting}
opt -load /path/to/libHello.so -load /path/to/libAnother.so -some-builtin-pass -another -hello a.bc -analyze -S
\end{lstlisting}

\subsection{References}
You can learn more about opt on the following websites

 - [http://llvm.org/docs/CommandGuide/opt.html](http://llvm.org/docs/CommandGuide/opt.html)
