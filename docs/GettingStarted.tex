\section{Getting Started}

\subsection{Prerequisites}
\begin{itemize}
    \item cmake (Beginner)
    \item C++ (intermediate)
    \item llvm (intermediate)
\end{itemize}

You can learn all the above from their respective documentation.

\subsection{Directory Structure}
Before starting to code, it is important to understand the directory structure of the library.

The library is divided into following directories:

\begin{itemize}
  \item cmake/      Important cmake modules required for building library
  \item docs/       Documentation
  \item include/    The public interface to the library
  \item samples/    Sample C code for testing and debugging
  \item src/        Implementation of the library
  \item tests/      Test code
  \item utils/      Small utility libraries
\end{itemize}

\subsection{Building a simple test pass}
In this section, we will discuss about building a small llvm pass using this
library. This pass takes the llvm module as input, then gathers points-to
constraints using library and then dumps them to the stdout.

Before moving on, you should have built a simple hello world llvm pass. If you
haven't, then follow the tutorial [here](http://llvm.org/docs/WritingAnLLVMPass.html).

You can build the pass inside the library in tests/ directory.

\subsection{Implementation}
First create a new file, say TestPass.cc, inside tests directory. Open the CMakeLists.txt file in the tests/ directory and append the following line to it.

\begin{lstlisting}
set(TEST_PASS_SOURCES TestPass.cc)
add_points_to_project(name ${TEST_PASS_SOURCES})
\end{lstlisting}

Now we can start writing the pass.

As we know that to write an llvm pass, we have to inherit from one of the ModulePass, BasicBlockPass, LoopPass, or FunctionPass. Since we are writing a pass that runs on the whole module, so we need to inherit from ModulePass.
The basic skeleton code looks like this,


\begin{lstlisting}
#include "llvm/Pass.h"
#include "llvm/Support/raw_ostream.h"
#include "llvm/IR/Function.h"

using namespace llvm;

namespace {

struct TestPass : public ModulePass {
    static char ID;  // required

    bool runOnModule(Module &module) {
        // our logic
    }
};

}

// static
char TestPass:ID = 0;

// registration of pass
static RegisterPass<TestPass> X("test-pass", "A short description",
                                false, // looks at CFG only
                                false  // doesn't modify anything
                                );
\end{lstlisting}

The logic required for this pass is simple. All we have to do is to generate
constraints for every instruction in the IR and save them in a data structure.
Then to show the output we can override the print method of ModulePass class.
In print method, we can just dump those constraints\newline
Now the next thing is to use *this* library for generating constraints.
Simplest way to use this library is to use the `ConstraintBuilder` class. This
class provides the method `getConstraint` with prototype as

\begin{lstlisting}
/**
 * \param instruction The LLVM instruction
 * \return Single Constraint object for single LLVM IR statement
 */
Constraint getConstraint(llvm::Instruction *instruction);
\end{lstlisting}

It takes an LLVM instruction pointer and returns a `Constraint` object. 
We can use `ConstraintBuilder` like,

\begin{lstlisting}
ConstraintBuilder builder;

for (auto &instruction : basicblock) {
    Constraint c = builder.getConstraint(&instruction);
    // not every instruction is a possible points-to constraint
    if (c != kInvalidConstraint) {
        continue;
    }

    // do whatever you like to do with c
}
\end{lstlisting}

The code looks like,

\begin{lstlisting}
#include "llvm/Pass.h"
#include "llvm/Support/ErrorHandling.h"
#include "llvm/IR/Function.h"
#include "llvm/IR/Instructions.h"
#include "llvm/Support/raw_ostream.h"

#include "ConstraintBuilder.h"

/* For storing the constraints */
#include <vector>

using namespace llvm;
using namespace ptsto;

namespace {

struct ConstraintGen : public ModulePass {
    static char ID;

    std::vector<Constraint> constraints_;

    void print(raw_ostream &os, const Module */**/) const override {
        for (const auto &constraint : constraints_) {
            constraint.dump(os);
        }
    }

    ConstraintGen() : ModulePass(ID) { }

    bool runOnModule(Module &module) override {
        ConstraintBuilder builder;

        for (Function &f : module) {
            for (BasicBlock &bb : f) {
                 for (Instruction &i : bb) {
                    Constraint c = builder.getConstraint(&i);
                    if (c == kInvalidConstraint) {
                        continue;
                    }

                    constraints_.push_back(c);
                }
            }
        }
        return false;
    }
};

}

// static
char TestPass:ID = 0;

// registration of pass
static RegisterPass<TestPass> X("test-pass", "A short description",
                                false, // looks at CFG only
                                false  // doesn't modify anything
                                );
\end{lstlisting}

